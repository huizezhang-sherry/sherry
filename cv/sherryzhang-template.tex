% Copyright (c) 2012 Cies Breijs
%
% The MIT License
%
% Permission is hereby granted, free of charge, to any person obtaining a copy
% of this software and associated documentation files (the "Software"), to deal
% in the Software without restriction, including without limitation the rights
% to use, copy, modify, merge, publish, distribute, sublicense, and/or sell
% copies of the Software, and to permit persons to whom the Software is
% furnished to do so, subject to the following conditions:
%
% The above copyright notice and this permission notice shall be included in
% all copies or substantial portions of the Software.
%
% THE SOFTWARE IS PROVIDED "AS IS", WITHOUT WARRANTY OF ANY KIND, EXPRESS OR
% IMPLIED, INCLUDING BUT NOT LIMITED TO THE WARRANTIES OF MERCHANTABILITY,
% FITNESS FOR A PARTICULAR PURPOSE AND NONINFRINGEMENT. IN NO EVENT SHALL THE
% AUTHORS OR COPYRIGHT HOLDERS BE LIABLE FOR ANY CLAIM, DAMAGES OR OTHER
% LIABILITY, WHETHER IN AN ACTION OF CONTRACT, TORT OR OTHERWISE, ARISING FROM,
% OUT OF OR IN CONNECTION WITH THE SOFTWARE OR THE USE OR OTHER DEALINGS IN THE
% SOFTWARE.

%%% LOAD AND SETUP PACKAGES

\usepackage[margin=0.75in]{geometry} % Adjusts the margins
\usepackage{array}
\usepackage{multicol} % Required for multiple columns of text

\usepackage{mdwlist} % Required to fine tune lists with a inline headings and indented content
\usepackage{enumitem} % For continuous enumeration
\usepackage{relsize} % Required for the \textscale command for custom small caps text

\usepackage{hyperref} % Required for customizing links
\usepackage{xcolor} % Required for specifying custom colors
\definecolor{dark-blue}{rgb}{0.15,0.15,0.4} % Defines the dark blue color used for links
\hypersetup{colorlinks,linkcolor={dark-blue},citecolor={dark-blue},urlcolor={dark-blue}} % Assigns the dark blue color to all links in the template
\usepackage{CJK}
%\usepackage{tgpagella} % Use the TeX Gyre Pagella font throughout the document
\usepackage[T1]{fontenc}
\usepackage{microtype} % Slightly tweaks character and word spacings for better typography
\usepackage{enumitem}

\pagestyle{empty} % Stop page numbering

%----------------------------------------------------------------------------------------
%	DEFINE MACROS
%----------------------------------------------------------------------------------------
\newcommand{\timeperiod}[2]{ % Specify start date and end date (optional) 
    \ifx&#2& % end date is empty
        #1%
    \else
        #1 -- #2%
    \fi
}
\newcommand{\zh}[1]{\begin{CJK}{UTF8}{bsmi}#1\end{CJK}}
\newcommand{\zhs}[1]{\begin{CJK}{UTF8}{gbsn}#1\end{CJK}}


%----------------------------------------------------------------------------------------
%	DEFINE STRUCTURAL COMMANDS
%----------------------------------------------------------------------------------------

\newenvironment{indentsection} % Defines the indentsection environment which indents text in sections titles
{\begin{list}{}{
    \setlength{\leftmargin}{\newparindent}
    \setlength{\parsep}{0pt}
    \setlength{\parskip}{0pt}
    \setlength{\itemsep}{0pt}
    \setlength{\topsep}{0pt}
    }
}
{\end{list}}

% name
\newcommand*\name[1]{ 
    \noindent{\LARGE\textbf{#1}}
    }

% Top level sections in the template, e.g. Education, Publications.
\newcommand*\roottitle[1]{
    \subsection*{#1}
    \vspace{-0.3em}
    \spacedhrule{0em}{0.5em}
    \nopagebreak[4]
    } 

% Section entry: Title, start date, end date, caption
\newcommand{\secentry}[4]{
    \noindent
    \begin{tabular}{@{} p{0.1\linewidth} p{0.87\linewidth} @{}} % @{} to remove left and right padding
        \timeperiod{#2}{#3} & \textbf{#1} \newline #4 % start date, end date, main text
    \end{tabular}
    \vspace{0.2cm}
}

% Simple section entry: start date, end date (optional), caption
\newcommand{\simpleentry}[3]{
    \noindent
    \begin{tabular}{@{}p{0.18\linewidth} p{0.8\linewidth}@{}} % @{} to remove left and right padding
        \timeperiod{#1}{#2} & #3             % start date, end date, caption
    \end{tabular}
    % \vspace{0.15cm}
}


% Talk entry: Entry, location, date
\newcommand{\twocolumnentry}[2]{
    \noindent
    \begin{tabular}{@{} p{0.1\linewidth} p{0.89\linewidth} @{}} % @{} to remove left and right padding
        \textit{#2} & #1 
    \end{tabular}
}

\newcommand{\twocolumnentries}[3]{
    \noindent
    \begin{tabular}{@{} p{0.1\linewidth} p{0.89\linewidth} @{}} % @{} to remove left and right padding
        \textbf{#2} & #1 
    \end{tabular}
}

% Section title, main text, date, subtext
\newcommand{\headedsection}[3]{
    \nopagebreak[4]
    \begin{indentsection}
        \item[]\textscale{1.1}{#1}\hfill#2#3
    \end{indentsection}
    \nopagebreak[4]
    } 

\newcommand{\talkentry}[3]{
    \noindent
    \begin{tabular}{@{} p{0.15\linewidth} p{0.61\linewidth} >{\raggedleft}p{0.20\linewidth} @{}} % @{} to remove left and right padding
        #3 & #1 & \textit{#2}
    \end{tabular}
}

\newcommand{\headedsubsection}[3]{
    \nopagebreak[4]
    \begin{indentsection}
        \item[]\textbf{#1}\hfill\emph{#2}#3
    \end{indentsection}
    \nopagebreak[4]
    } % Section title used for a new position

\newcommand{\bodytext}[1]{\nopagebreak[4]\begin{indentsection}\item[]#1\end{indentsection}\pagebreak[2]} % Body text (indented)

\newcommand{\inlineheadsection}[2]{
    \begin{basedescript}
        {\setlength{\leftmargin}{\doubleparindent}}
        \item[\hspace{\newparindent}\textbf{#1}]#2
    \end{basedescript}
    } % Section title where body text starts immediately after the title

\newenvironment{indentpar}[1]{  % Environment for entire indented paragraph 
    \begin{list}{}
        {\setlength{\leftmargin}{#1}}
            \item[]
    }
    {\end{list}}

\newcommand*\acr[1]{\textscale{.85}{#1}} % Custom acronyms command

\newcommand*\bull{\ \ \raisebox{-0.365em}[-1em][-1em]{\textscale{4}{$\cdot$}} \ } % Custom bullet point for separating content

\newlength{\newparindent} % It seems not to work when simply using \parindent...
\addtolength{\newparindent}{\parindent}

\newlength{\doubleparindent} % A double \parindent...
\addtolength{\doubleparindent}{\parindent}

\newcommand{\breakvspace}[1]{\pagebreak[2]\vspace{#1}\pagebreak[2]} % A custom vspace command with custom before and after spacing lengths
\newcommand{\nobreakvspace}[1]{\nopagebreak[4]\vspace{#1}\nopagebreak[4]} % A custom vspace command with custom before and after spacing lengths that do not break the page

\newcommand{\spacedhrule}[2]{\breakvspace{#1}\hrule\nobreakvspace{#2}} % Defines a horizontal line with some vertical space before and after it